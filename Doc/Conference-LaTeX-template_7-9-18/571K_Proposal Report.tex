\documentclass[conference]{IEEEtran}
\IEEEoverridecommandlockouts
% The preceding line is only needed to identify funding in the first footnote. If that is unneeded, please comment it out.
\usepackage{cite}
\usepackage{amsmath,amssymb,amsfonts}
\usepackage{algorithmic}
\usepackage{graphicx}
\usepackage{textcomp}
\usepackage{xcolor}
\def\BibTeX{{\rm B\kern-.05em{\sc i\kern-.025em b}\kern-.08em
    T\kern-.1667em\lower.7ex\hbox{E}\kern-.125emX}}
\begin{document}

\title{Load Balancing Mechanism Based on Ant Colony Optimization for Smart-City Task Scheduling in IoT\\
{
}
}

\author{\IEEEauthorblockN{Xiaojun Tian}
\IEEEauthorblockA{\textit{Department of Computer Engineering} \\
\textit{University of British Columbia}\\
Vancouver, Canada \\
xtian01@ece.ubc.ca}
\and
\IEEEauthorblockN{Yixuan Ji}
\IEEEauthorblockA{\textit{Department of Computer Engineering} \\
\textit{University of British Columbia}\\
Vancouver, Canada \\
jiyixuan@ece.ubc.ca}
}

\maketitle

\section{Introduction}
\subsection{Overview}
IoT is the foundation of future smart-city. The smart-city is the idea that uses multi data collection pieces of equipment to supply information which is used to manage assets and resources efficiently. This huge amount of data is processed and analyzed to monitor and manage smart-city function parts like water supply networks, transportation systems, hospitals, schools, and many other public community services. However, the difficulty in the operation of the smart-city might not just count on the processing ability of the IoT terminal, one big existing issue is how to allocate different tasks to the different terminals based on each processing ability and make the whole smart-city more efficient. Here, we introduce cloud-computing to take charge of the task scheduling of the smart-city based on its powerful processing and computing power.

Cloud computing is a hot technology in the IoT area and experiencing a rapid development both in academia and industry at the same time. The cloud computing contains distributed computing, parallel computing, and grid computing. This technology aims to offer distributed, virtualized, and elastic resources as utilities to end clients. It has the potential to support the full realization of ‘computing as a utility’ in the near future \cite{b1}. It is not an easy thing for us to manually assign tasks to computing resources\cite{b2}\cite{b3}, which also represents the smart-city IoT terminal in clouds. It is very important to find an appropriate scheduling algorithm to allocate massive tasks into cloud computing resources and raise the reliability of the smart-city and IoT system. Cloud task scheduling essentially is an NP-hard optimization problem, and many algorithms have been proposed to improve its performance.

In computing, load balancing improves the distribution of workloads across multiple computing resources, such as computers, a computer cluster, network links, central processing units, or disk drives\cite{b4}. Load balancing is indispensable for cloud task scheduling. Firstly, the cloud computing must use load balancing in its own platform to provide a solution with high efficiency for the client; Secondly, load balancing mechanism is needed to construct a low cost and infinite resource pool for the client.

Ant colony optimization algorithm (ACO) is a biological simulation originally came from the process of ant shortest tracks finding in nature. Ants could leave a substance called ‘pheromone’ in their campaign path. The other ants in the cluster can sense this kind of ‘pheromone’ and follow the relatively shorter path each time. Ant colony was formed by a large number of collective behavior and this would show a positive feedback ‘phenomenon’ of information. After the positive filtered by ant colony, one optimal path, the optimization solution which was passed by the most number of ants would later be chosen.

What we gonna do: 
\\In order to improve the performance of cloud task scheduling in smart-city. Here we want to bind the load balancing mechanism with the ant colony optimization algorithm to address the cloud scheduling issue. Many works have been done in this area from different perspectives so far actually. What we gonna  do in this course project is firstly implement basic ant colony optimization algorithm\cite{b5} in JavaScript; Secondly, realize two improved load balancing ant colony optimization algorithms in the paper\cite{b6}\cite{b7} also in JavaScript; Thirdly, simulate the real cloud task scheduling situation in CloudSim toolkit package, a framework for modeling and simulation of cloud computing infrastructures and services, and make the performance comparison with basic ant colony optimization algorithm and two improved algorithms. At last, we put these three algorithms into real cloud application like Apache Hadoop or Google Cloud to see and compare each one of their performance.
\subsection{Evaluation}
Here we have two methods to evaluate our approaches:

The first method is to see the convergence of ant colony optimization algorithm. The ant colony optimization algorithm is actually a learning process. Therefore, the total makespan of the tasks will be convergence to a stable level after iterations. We will set a different appropriate iteration threshold for each algorithm based on our prior knowledge. If the iteration number exceeds the threshold, we treat it is a failure algorithm, vice versa.

The second method is to see the total makespan of each algorithm. Even, the algorithm meets the first evaluation criterion, if the total makespan in a real cloud scheduling situation within our tolerance level, we treat it is a success algorithm, vice versa. We will also set a different threshold for each real situation.

The algorithm needs to satisfy two criteria above, we think the approach is successful.

\section{Related Work}

An individual ant is a quite random and unordered insect in the world of animals, which shows us a quite similar characteristic to a piece of data stream. From a microscopical perspective, the behavior of the group of ant has a more stable and regular pattern as people can observe. Furthermore, many meaningful tasks can be implemented by the ant colony rather than the individual ant, such as foraging (searching for food), building nests \cite{b8}\cite{b9}, moving material and so on. Always, the ant colony have the potential of optimizing a very complex problem by randomly exploring, modifying and deciding. This a collective behavior could be commonly seen from a group of social insects \cite{b10}, which is “swarm intelligence” \cite{b11}. 

The ant colony algorithm from nature have been applied into many scenarios in load balancing mechanisms. One of the mechanisms was proposed by the Zehua Zhang and Xuejie Zhang. This mechanism is based on a combination of ant colony algorithm and complex network theory and specifically designed for open cloud computing federation \cite{b7}. Their work was realized in an agent-based system, where each task can be deemed as an ant. A lot of cloud computing service provider’s will always result in many management regions and those regions could be viewed as a complex network. Each task starts at a node of the network periodically, the route it walked would have pheromone. The existing load on the network will influence the path of ant colony. Also, the pheromone left by ant would indicate the path the rest should go. An ant can remember the overload path and update it by leaving pheromone \cite{b7}. This mechanism is quite useful in handling complex network in load balancing issues.

There is a Load Balancing Ant Colony Optimization algorithm aiming at finding optimal resource allocation for tasks in dynamical cloud system \cite{b6}. It can balance the whole system while reducing the average running time for tasks. The simulation is based on CloudSim which offers hosts, virtual machines, applications and so on. In this work, we view all VMs as different nodes with random initial pheromones. Then let the ants begin at starting nodes to propagate and update pheromone after each iteration. Measure the makespan for tasks and calculate total execution time of tasks in the end. In this case, a well-balanced load across Cloud and nodes should be achieved.

With network model and optimization ant colony algorithm, we are going to combine the advantages of ant colony algorithm and apply it into smart-city to solve similar issues. Furthermore, comparison will be made to check performance based on the simulation results for smart-city.


\begin{thebibliography}{00}
\bibitem{b1}  A. Weiss, ``Computing in the Clouds,'' in  netWorker on Cloud computing: PC functions move onto the web, vol. 11, 2007, pp. 16--25.
\bibitem{b2} F. Chang, J. Ren and R. Viswanathan, ``Optimal resource allocation for batch testing,'' in ICST, 2009 IEEE International Conference on Software Testing Verification and Validation, pp.91-100, 2009.
\bibitem{b3} F. Chang, J. Ren, and R. Viswanathan, “Optimal Resource Allocation in Clouds” in 2010 IEEE 3rd International Conference on Cloud Computing, pp.418-425, 2010.
\bibitem{b4} Iqbal M A, Saltz J H, Bokhart S H. Performance tradeoffs in static and dynamic load balancing strategies[J]. 1986.
\bibitem{b5} M. Dorigo, L.M. Gambardella, “Ant colony system: A cooperative learning approach to the traveling salesman problem”, in IEEE Transactions on Evolutionary Computation (1997), DOI:10.1109/4235.585892, pp.53–66, 1997.
\bibitem{b6}Li, Kun, et al. "Cloud task scheduling based on load balancing ant colony optimization." 2011 Sixth Annual ChinaGrid Conference. IEEE, 2011.
\bibitem{b7} Zhang, Zehua, and Xuejie Zhang. "A load balancing mechanism based on ant colony and complex network theory in open cloud computing federation." Industrial Mechatronics and Automation (ICIMA), 2010 2nd International Conference on. Vol. 2. IEEE, 2010.
\bibitem{b8} E.Bonabeau, MDorigo, and G.Theraulaz, "Inspiration for
optimization from social insect behavior," Nature, vo1.406, pp.39-
42, July2000.

\bibitem{b9} MDorigo, G.D.Caro, and L.M.Gambardella, "Ant algorithms for
discrete optimization," ArtifLife, vol.5, no.2, pp.137-172, 1999.

\bibitem{b10} R. Parpinelli, H. Lopes and A. Freitas, "Data mining with an ant colony optimization algorithm", IEEE Transactions on Evolutionary Computation, vol. 6, no. 4, pp. 321-332, 2002. Available: 10.1109/tevc.2002.802452 [Accessed 29 January 2019].
\bibitem{b11} E. Bonabeau, M. Dorigo, and G. Theraulaz, Swarm Intelligence: From
Natural to Artificial Systems. New York: Oxford Univ. Press, 1999.

\end{thebibliography}

\end{document}
